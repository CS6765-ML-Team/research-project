\begin{document}

\section{History of Machine Learning in Medicine}

The medical industry revolves around the collection of data and using that data to make diagnoses to save lives. As such, the application of a tool like machine learning, where learning from data is the focus, is clear in such an industry where data collection is so incredibly important. The history of artificial intelligence and machine learning in medicine spans many decades. Periods of rapid innovation followed by periods of AI "Winters" define this history; however, it begins with the centralization and digitization of medical data that would help propel the research and development of the first prototype medical AI systems.

\subsection{A Medical Library for Machines}

During the beginning of research into AI's potential application in the field of medicine, many issues were raised in regards to how the AI models could be used effectively. The two primary issues were 1) how to connect electronic health records and clinical documentation from around the world, and 2) how can a machine understand the underlying knowledge required for operation in a field as complex and critical as medicine~\cite{kulikowski_beginnings_2019}. Work towards solving these problems began in the 1960's at the National Library of Medicine (NLM), where the Medical Literature Analysis and Retrieval System (MEDLARS), and its web-based search engine, PubMed, would be developed and provide access to the world's largest repository of biomedical literature~\cite{kulikowski_beginnings_2019}. This, alongside the NLM's support of capturing and computationally representing medical terminologies and vocabularies~\cite{kulikowski_beginnings_2019}, essentially solved both of the aforementioned problems. There was now access to the medical data required to train an AI system, and there now existed a computational representation of medical vocabulary and knowledge required for an AI system to confidently operate in the field. It should be noted that the development of these tools was not specifically done for the advancement of AI in medicine; nonetheless, they would prove to be essential in the decades to come.

\subsection{The First AI Winter}

What followed in the 1970's is what many refer to as the first "AI Winter", where there was a significant decrease in interest and investment into the field of AI, not just AI in medicine. Many believe the cause of this first winter to be due to the perceived limitations of artificial intelligence at the time~\cite{kaul_history_2020}, forcing people to disregard the field when in reality it was still in its infancy. Despite this deceleration in research, the 1970's still represented a major stepping stone in the use of AI in medicine. This was accomplished through collaboration between universities across the United States and facilitated by an inter-university time-shared computer system based at Stanford called SUMEX-AIM (Stanford University Medical Experimental - AI in Medicine) developed in 1973~\cite{kulikowski_beginnings_2019}. This collaboration allowed for discussions on many of the concerns that AI's involvement in medicine presented at the time, as well as provided a breeding ground for innovative approaches to the newly developing field. 

\subsection{The First Successful Prototype Systems}

Throughout and following the collaborative innovation of the 1970's was the development of three majorly successful prototype AI systems used for medical diagnosis. These three systems, which will be explained and discussed in this section, provided an idea of what was possible with AI and laid the groundwork for AI's future involvement in the medical field.

\subsubsection{MYCIN}

MYCIN, developed in the 1970's, was a rule-based AI system that used a knowledge-base of roughly 600 rules~\cite{kaul_history_2020} to identify infectious diseases a patient may have and then suggest antibiotic treatment tailored to the patient. Physicians would interface directly with the system by inputting the patient's information and MYCIN would output a list of potential infections, as well as their respective treatments based on information provided about the patient. The list produced by MYCIN was achieved through backward chaining, where MYCIN would start with a given infection and work backwards to try to match it to the patient's symptoms, and used a confidence factor representation to measure clinical uncertainty between the potential diagnoses~\cite{kulikowski_beginnings_2019}. Although this inference method and confidence factor is akin to calculating a probability for a given diagnosis being the correct one, the confidence-factor representation was actually found to psychologically aid in accepting MYCIN consultations~\cite{kulikowski_beginnings_2019}. MYCIN demonstrated that early AI systems had the potential to seriously improve the diagnosis process in medicine, even if only confined to hyper-specific subdomains of the field. Dr. Casimir Kulikowski, one of the authors of CASNET, went on to state that MYCIN "was the most influential expert system that demonstrated the power of modularized rules for representing decision-making"~\cite{kulikowski_beginnings_2019}. The work done on MYCIN and the results it demonstrated would go on to encourage researchers in the field to create more AI prototype systems to aid in diagnosis within different subdomains of medicine, Dr. Kulikowski and his system CASNET being the next one.

\subsubsection{CASNET}

The Causal Associational Network (CASNET) was a consultation program designed specifically for glaucoma patients developed at Rutgers University in the 1970's. CASNET used causal explanations of disease, alongside empirical knowledge of presumptive diagnoses, prognoses, and treatments to advise glaucoma patients~\cite{kulikowski_beginnings_2019}. The CASNET model was divided into three programs~\cite{kaul_history_2020}: model-building, consultation, and a maintained database. By applying disease specific information to a patient of interest, CASNET could provide physicians with patient specific care advice. CASNET was unveiled at the Academy of Ophthalmology meeting in Las Vegas, Nevada in 1976~\cite{kaul_history_2020}.

\subsubsection{INTERNIST-I}

Whereas MYCIN and CASNET were designed to be applied in only small, hyper-specific subdomains of medicine, INTERNIST-I was a system developed to cover a broad range of different medical diagnoses within internal medicine. Developed at the University of Pittsburgh by AI researcher Harry Pople and medicine specialist Dr. Jack Myers~\cite{kulikowski_beginnings_2019}, INTERNIST-I was a prototype AI system designed to assist in general internal medicine diagnosis by modeling the expertise of Dr. Myers~\cite{wolfram_appraisal_1995}. Due to the estimated size of the internal medicine field and the number of diseases it contains, designing a system that could confidently classify symptoms and provide correct diagnoses was a monumental task. Development of INTERNIST-I and its accompanying knowledge-base began in the 1970's, and by 1988 the knowledge-base had grown to contain roughly 600 diseases and represented about 25 person-years of effort~\cite{wolfram_appraisal_1995}. This massive knowledge-base would mean INTERNIST-I would have difficulty with complex diagnosis, when there wasn't a single, clear answer. Despite this drawback, INTERNIST-I would demonstrate that the diagnosis process could be broken down into a pure classification problem, instead of the probabilistic models used by MYCIN and CASNET. As such, INTERNIST-I is considered a crucial step in the development of early AI systems in medicine.

\subsection{The First AI Algorithms}

The advent of AI in medicine provided a training ground to validate the usefulness of the different AI algorithms that had already been established at the time. By observing the impact a training algorithm has in a complicated field such as medicine, researchers would be able to judge how useful the algorithm could be in the greater AI landscape. Focusing only on medical diagnosis at the time of the first AI prototype systems discussed above, there existed two primary AI algorithms being employed: decision trees and Bayesian classification.  

\subsubsection{Decision Trees}
In a medical diagnosis, it is up to the diagnostician to use the patient's symptoms to iteratively rule out possible causes until the final diagnosis is reached. As such, the use of decision trees and decision rules to narrow down a medical diagnosis makes intuitive sense, and early researchers agreed with this. Decision trees were considered to be the most promising area for medical data analysis within a potential AI system~\cite{kononenko_machine_2001}. This promise was further amplified with the invention of the Iterative Dichotomizer 3 (ID3) decision tree algorithm and its application in oncological diagnosis~\cite{kononenko_machine_2001}. Decision trees continued to be a popular option for medical diagnosis due to their more easily interpretable structure. This allowed for the algorithm to better explain its inductive reasoning, re-assuring the physician and the patient on the diagnosis~\cite{kononenko_machine_2001}. 

\subsubsection{Bayesian Classification}
Bayesian classification is a probabilistic approach to inference that uses Bayes' theorem to calculate how likely a given hypothesis is. This naturally maps to the world of medical diagnosis, as given a list of patient symptoms, a Bayesian model assigns probabilities to each potential diagnosis, similar to a physician. Although the transparency, or ability to explain its reasoning through its representation, of Bayesian classification was a cause for concern early on, it was found that using Bayesian classifiers far outperformed other AI algorithms in medical diagnostic tests~\cite{kononenko_machine_2001}. With improvements to the issue of transparency addressed in the 1990's~\cite{kononenko_machine_2001}, Bayesian classification became a commonly used approach in the realm of AI medical diagnosis systems.  

\newpage

\end{document}