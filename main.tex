%%%% ijcai19-multiauthor.tex

\typeout{IJCAI-19 Multiple authors example}

% These are the instructions for authors for IJCAI-19.

\documentclass{article}
\pdfpagewidth=8.5in
\pdfpageheight=11in
% The file ijcai19.sty is NOT the same than previous years'
\usepackage{ijcai19}

% Use the postscript times font!
\usepackage{times}
\usepackage{soul}
\usepackage{url}
\usepackage[hidelinks]{hyperref}
\usepackage[utf8]{inputenc}
\usepackage[small]{caption}
\usepackage{graphicx}
\usepackage{amsmath}
\usepackage{booktabs}
\urlstyle{same}

% the following package is optional:
%\usepackage{latexsym} 

\bibliographystyle{named} % We choose the "plain" reference style


% Following comment is from ijcai97-submit.tex:
% The preparation of these files was supported by Schlumberger Palo Alto
% Research, AT\&T Bell Laboratories, and Morgan Kaufmann Publishers.
% Shirley Jowell, of Morgan Kaufmann Publishers, and Peter F.
% Patel-Schneider, of AT\&T Bell Laboratories collaborated on their
% preparation.

% These instructions can be modified and used in other conferences as long
% as credit to the authors and supporting agencies is retained, this notice
% is not changed, and further modification or reuse is not restricted.
% Neither Shirley Jowell nor Peter F. Patel-Schneider can be listed as
% contacts for providing assistance without their prior permission.

% To use for other conferences, change references to files and the
% conference appropriate and use other authors, contacts, publishers, and
% organizations.
% Also change the deadline and address for returning papers and the length and
% page charge instructions.
% Put where the files are available in the appropriate places.

\title{Application of Artificial Intelligent in Healthcare}

\author{
Ethan Garnier$^1$\footnote{Contact Author}\and
Matthew Tidd$^2$\And
Minh Nguyen$^{2}$
\affiliations
$^1$Electrical and Computer Engineering Department, UNB\\
$^2$Mechanical Engineering Department, UNB
\emails
\{ethan.garnier78, matthew.tidd, mnguyen6\}@unb.ca
}

\begin{document}

\maketitle

\begin{abstract}
This short example shows a contrived example on how to format the authors' information for {\it IJCAI--19 Proceedings} using \LaTeX{}.
\end{abstract}

\section{Introduction}

This short example shows a contrived example on how to format the authors' information for {\it IJCAI--19 Proceedings}.
\section{State of the art}

\section{Challenges}
\subsection{Trust}
\subsection{Accountability}
\subsection{Data privacy and protection}
AI technologies depend on a vast amount of patient data and record to make accurate predictions when subjected to unseen data.
However, in the event of a database violation or data breach, confidential patient information can be exploited for malicious intentions (identity theft, social stigma, discrimination,...).
This can put a mental burden on the patient suceptible to the violation and other relevant stakeholders.
Therefore, adequate law and guidelines are crucial to regulate the application of AI in medical and prevent misuse of data


The data gathering and handling of health data in the US is controversial, raising legal and ethical privacy questions \cite{price_privacy_2019}.
Although the data can originate from various sources, such as healthcare providers, insurance claim, and wearable devices, US privacy law operates in different extents depending on the data source.
This law also depends on the custodian of the data. Under the Health Insurance Portability and Accountability Act (HIPAA), the federal Privacy Rule only governs data handling between the conventional entities, such as healthcare providers, health insurance provides, patients, and intermediaries. 
However, \cite{price_privacy_2019} pointed out existing gap in HIPAA regulation.
Although, HIPAA protect patient privacy from health data breach via a deindentifying process, patient data can be reidentified through data triangulation from other datasets.


Furthermore, a more fundamental problem is the amount of health related data that is not regulated under HIPAA. Originally enacted to regulate data privacy in health records and between covered entities, HIPAA did not account for tech companies that are developing smart wearable devices

\section{Conclusion}

\section{Template notes}

\subsection{Author names}

Each author name must be followed by:
\begin{itemize}
    \item A newline {\tt \textbackslash{}\textbackslash{}} command for the last author.
    \item An {\tt \textbackslash{}And} command for the second to last author.
    \item An {\tt \textbackslash{}and} command for the other authors.
\end{itemize}

\subsection{Affiliations}

After all authors, start the affiliations section by using the {\tt \textbackslash{}affiliations} command.
Each affiliation must be terminated by a newline {\tt \textbackslash{}\textbackslash{}} command. Make sure that you include the newline on the last affiliation too.

\subsection{Mapping authors to affiliations}

If some scenarios, the affiliation of each author is clear without any further indication (\emph{e.g.}, all authors share the same affiliation, all authors have a single and different affiliation). In these situations you don't need to do anything special.

In more complex scenarios you will have to clearly indicate the affiliation(s) for each author. This is done by using numeric math superscripts {\tt \$\{\^{}$i,j, \ldots$\}\$}. You must use numbers, not symbols, because those are reserved for footnotes in this section (should you need them). Check the authors definition in this example for reference.

\subsection{Emails}

This section is optional, and can be omitted entirely if you prefer. If you want to include e-mails, you should either include all authors' e-mails or just the contact author(s)' ones.

Start the e-mails section with the {\tt \textbackslash{}emails} command. After that, write all emails you want to include separated by a comma and a space, following the same order used for the authors (\emph{i.e.}, the first e-mail should correspond to the first author, the second e-mail to the second author and so on).

You may ``contract" consecutive e-mails on the same domain as shown in this example (write the users' part within curly brackets, followed by the domain name). Only e-mails of the exact same domain may be contracted. For instance, contracting ``person@example.com" and ``other@test.example.com" is not allowed because the domains are different.


% Bibliography/Reference Stuff
\bibliography{main}
\end{document}